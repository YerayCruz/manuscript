\thispagestyle{empty} % No headers or page numbers

\begin{center}
\large { Instituto Tecnológico y de Estudios Superiores de Monterrey\\}
\vspace*{0.5em}
\normalsize { Campus Monterrey } \\
\vspace*{0.5em}
\large{School of Engineering and Sciences\\}
\vspace*{1em}

\begin{figure}[!h]
 \begin{center}
  \includegraphics[scale=0.08]{pics/LogoTec.pdf}
 \end{center}
\end{figure}
\vspace*{0.6cm}

\Large {Induced Rotation of Ratchets in Driven Environments}

\vspace*{0.5em}
\normalsize{A thesis presented by}\\
\vspace*{0.5em}
\large{Yeray Cruz Ruiz}\\
\vspace*{1.5cm}
\normalsize

Submitted to the\\
School of Engineering and Sciences\\
in partial fulfillment of the requirements for the degree of\\
\vspace*{1em}
\large{Master} \\
\large{In} \\
\large{Nanotechnology} \\
\normalsize
\vfill
Monterrey, Nuevo León, México, Dec 2025\\
\end{center}

\newpage %-------------------------------------------- -----------------PAGE iii

\begin{center}

\large { Instituto Tecnológico y de Estudios Superiores de Monterrey\\}
\normalsize {Campus Monterrey}
\normalsize {School of Engineering and Sciences}
\vspace*{1em}

The commitee members, hereby, certify that have read the thesis presented by {\bf Yeray Cruz Ruiz} and that it is fully adequate in scope and quality as a partial requirement for the degree of Master in Nanotechnology.\\
\vspace*{3em}

\raggedleft{
{---------------------------------} \\
Antonio Ortiz Ambriz, PhD \\
Tecnológico de Monterrey\\
School of Engineering and Sciences\\
Principal Advisor\\
\vspace*{2em}
{---------------------------------}\\
Servando López Aguayo, PhD  \\
Tecnológico de Monterrey\\
School of Engineering and Sciences\\
Committee member  \\
\vspace*{2em}
{---------------------------------}\\
Helena Massana Cid, PhD  \\
Universitat de Barcelona\\
Departament de Física de la Matèria Condensada\\
Committee member  \\
\vspace*{2em}
}

\center{
\vfill
{--------------------------------------------------} \\
Dr. Elisa Virginia Vázquez Lepe\\
Dean of Graduate Studies\\
School of Engineering and Sciences}\\
Monterrey, Nuevo León, México, Dec 2025
\end{center}

\newpage %------------------------------------------PAGE iv

\begin{center}
\bf{Declaration of Authorship}
\end{center}

I, Yeray Cruz Ruiz (student), declare that this thesis titled, \enquote{Induced Rotation of Ratchets in Driven Environments} and the work presented in it are my own. I confirm that:

\begin{itemize}
  \item This work was done wholly or mainly while in candidature for a research degree at this University.
  \item Where any part of this thesis has previously been submitted for a degree or any other qualification at this University or any other institution, this has been clearly stated.
  \item Where I have consulted the published work of others, this is always clearly stated.
  \item Where I have quoted from the work of others, the source is always given. Apart from such quotations, this thesis is entirely my own work.
  \item I have acknowledged all main sources of help.
  \item Where the thesis is based on work done by myself jointly with others, I have made clear exactly what was done by others and what I have contributed myself.
\end{itemize}

\vspace{5em}

\begin{flushright}
{-------------------------------------------------} \\
Yeray Cruz Ruiz (Student) \\
Monterrey, Nuevo León, México, Dec 2025
\end{flushright}

\vfill
\begin{center}
\large {\copyright 2025 by Yeray Cruz Ruiz}\\
\large{All Rights reserved}
\end{center}


\newpage %------------------------------------------PAGE iv

\Huge {\bf Dedication}

\vfill
\normalsize For those who come after. 
\vfill


\newpage %------------------------------------------PAGE v 
\Huge {\bf Acknowledgements} \\
\vfill
\normalsize 
I thank SECIHTI and TEC de Monterrey for their scholarship, letting me pursue a Master's degree.

I am deeply appreciative of Dr. Helena and Dr. Servando for the time and effort they devoted to reviewing this manuscript, ensuring that it meets the highest standards.

To my parents and my brother, your unwavering support has been a constant source of strength. Thank you for always being there, especially during the most challenging moments of this journey.

A heartfelt thanks to my friends, 
David Correa,
Francisco Vázquez,
Leonardo Alanis,
and Shay Farías
who have become part of my family, made this journey easier and more enjoyable than it had to be. 

Finally I am especially thankful to Dr. Antonio, your belief in me, combined with your guidance as both a mentor and a friend, has been invaluable. Thank you for inspiring me. 
\vfill

\vspace{5em}

\newpage %------------------------------------------PAGE vii
\begin{center}
  \large {{\bf Induced Rotation of Ratchets in Driven Environments}\\
  by\\
Yeray Cruz Ruiz}
\end{center}
\addcontentsline{toc}{chapter}{Abstract}
\Huge{{\bf Abstract}} \\
\normalsize
\vspace*{0em}

Swimming at the mesoscale has been a topic of growing interest over the past two decades due to the unique dynamics that arise at low Reynolds numbers, a regime where viscous forces dominate over inertial ones. In this environment, motion becomes nonintuitive: as described by the scallop theorem, a swimmer with only one degree of freedom cannot achieve net displacement because its movement is time-reversible under the Navier–Stokes equations.
Despite these physical constraints, nature has evolved remarkable strategies to overcome them. Many microorganisms exhibit nonreciprocal motion, allowing them to propel themselves even in highly viscous media. For instance, \textit{Escherichia Coli} uses a rotating flagellum to push fluid backward, propelling the bacterium forward.
Inspired by such biological systems, researchers have sought to engineer artificial swimmers capable of movement under similar conditions. One notable example involves bacterial micromotors, where a ratchet-shaped structure is immersed in a bacterial bath. The bacteria, moving along ballistic trajectories, collide with the asymmetric surface of the ratchet, transferring momentum that induces rotation. Over time, however, this motion ceases as the bacteria exhaust the nutrients in their medium, requiring replenishment to sustain activity.
This form of active matter relies on energy extracted from its environment. Yet a natural question arises: can we achieve similar directed motion using anisotropically driven—but externally controlled—systems?

We study the dynamics of paramagnetic colloids driven by an externally applied precessing magnetic field, which forces the particles’ magnetic moments to rotate. The colloids are confined vertically between two glass plates and the system implements periodic boundary conditions along the \textit{x}- and \textit{y}-directions. Under these conditions, dipole–dipole interactions emerge, producing attraction or repulsion depending on the instantaneous orientation of the rotating magnetic moments. At low driving frequencies, these interactions cause the particles to form rotating dimers that share a common center of mass. In contrast, in the intermediate-frequency range (3.25–7 Hz), the particles frequently break and reform pairs, undergoing a neighbor-exchange process that results in ballistic trajectories. To examine whether this collective motion can perform mechanical work, we introduce an asymmetric ratchet-like structure into the system. At the low-frequency dimer-rotation regime, the ratchet exhibits negligible angular velocity, whereas in the neighbor-exchange regime, its rotation cannot be attributed solely to thermal fluctuations.

\newpage %-------------------------------------------PAGE viii

\listoffigures
\addcontentsline{toc}{chapter}{List of Figures}

\listoftables
\addcontentsline{toc}{chapter}{List of Tables}

% \listoffigures
% \listoftables
\tableofcontents
\newpage

\pagenumbering{arabic} % Change page numbering back to Arabic numerals
