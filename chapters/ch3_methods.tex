\part{Methods}

\chapter{Simulations}

Numerical simulations provide a powerful framework to reproduce and analyze physical phenomena under controlled conditions. Unlike experiments, they allow for precise manipulation of initial parameters, systematic testing of hypotheses, and straightforward replication of results. In particular, simulations are invaluable when dealing with microscopic systems, where stochastic effects and complex interactions can make experimental observations challenging or ambiguous. 

In this thesis, numerical simulations are employed to study the dynamics of colloidal systems at the microscale. Such systems are often dominated by thermal fluctuations and many-body interactions, making them difficult to probe experimentally without advanced imaging and data analysis techniques. The primary computational approach used here is \textit{molecular dynamics} (MD), which will be described in detail in the following sections. MD enables the integration of particle trajectories under the influence of deterministic and stochastic forces, providing direct access to observables such as mean-squared displacements, velocity correlations, and transport coefficients.



\section{Molecualr dynamics}

\section{Data analysis}
