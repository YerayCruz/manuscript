\part{Methods}

\chapter{Simulations}

Numerical simulations provide a powerful framework to reproduce and analyze physical phenomena under controlled conditions. Unlike experiments, they allow for precise manipulation of initial parameters, systematic testing of hypotheses, and straightforward replication of results. In particular, simulations are invaluable when dealing with microscopic systems, where stochastic effects and complex interactions can make experimental observations challenging or ambiguous. 

In this thesis, numerical simulations are employed to study the dynamics of colloidal systems at the microscale. Such systems are often dominated by thermal fluctuations and many-body interactions, making them difficult to probe experimentally without advanced imaging and data analysis techniques. The primary computational approach used here is \textit{molecular dynamics} (MD), which will be described in detail in the following sections. MD enables the integration of particle trajectories under the influence of deterministic and stochastic forces, providing direct access to observables such as mean-squared displacements, velocity correlations, and transport coefficients.

\section{Molecualr dynamics}

This thesis explores the physics of the system by using molecular dynamics, where randomness enters through thermal noise. As we saw in \ref{stochasticrepresentation}, the interactions can be derived from classical mechanics. In this case, Newton’s second law gives the starting point to calculate the forces acting on a single particle. In other words, at each iteration of the simulation we need to evaluate the forces that each particle experiences. Taking equation~(\ref{eq:newton}) as a starting point, and adding labels for the different interactions, we obtain:

\begin{equation}
  m_i\ddot{\vec{x_i}} = F^{collision}_i + F^{drag}_i + \eta(t)\text{,}
  \label{eq:langevinratchet}
\end{equation}

where $m_i$ is the mass of particle $i$, $\vec{x_i}$ its position, and $F^{drag}_i$ is the viscous force, written as $-\lambda \dot{\vec{x}}_i$, with $\lambda$ the drag coefficient of the fluid. Since we are working at a scale where inertial effects are negligible, the term $m_i\ddot{\vec{x_i}}$ can be dropped. The random force $\eta(t)$ represents collisions with the solvent particles. In this thesis, $\eta$ is taken as a Gaussian random force with $\expval{\eta} = 0$ and correlation $\expval{\eta_i(t)\eta_j(t')} = 2k_B T \lambda \delta_{i,j}\delta(t-t')$. This form will be used for non-paramagnetic particles.

For paramagnetic particles, we need to add the contribution of dipole–dipole interactions, which gives the modified equation:

\begin{equation}
  m_i\ddot{\vec{x_i}} = F^{collision}_i + F^{drag}_i + F^{dd}_i + \eta(t)\text{,}
  \label{eq:langevindipole}
\end{equation}

where $F^{dd}_i$ according to \cite{yung1998analytic} is described as force exerted for the dipole moment $\vec{m}_i$ on a dipole $\vec{m}_j$:

\begin{equation}
  \label{eq:dipoledipoleforce}
\vec{F}^{dd}_i = \frac{3\mu_0}{4\pi r^4}
\begin{multlined}[t]
\bigl[ (\hat{x}_{i,j} \times \vec{m}_i) \times \vec{m}_j
    + (\hat{x}_{i,j} \times \vec{m}_j) \times \vec{m}_i \\
    - 2\hat{x}_{i,j}(\vec{m}_i \cdot \vec{m}_j)
    + 5\hat{x}_{i,j}((\hat{x}_{i,j} \times \vec{m}_i) \cdot (\hat{x}_{i,j} \times \vec{m}_j)) \bigr],
\end{multlined}
\end{equation}

where $\hat{x}_{i,j}$ is the unitary vector of distance between dipoles. Since all magnetic dipoles are identical and aligned with the external magnetic field $\vec{B}_{\text{ext}}$, we can express the dipole moment as $\vec{m} = \chi V \vec{H}_{\text{ext}} = \frac{\chi V}{\mu_0} \vec{B}_{\text{ext}}$, where $\chi$ is the magnetic susceptibility and $V$ is the particle volume. Substituting $\vec{m}_i = \vec{m}_j = \vec{m}$ and expressing in terms of the external field gives:

\begin{equation}
  \label{eq:dipoledipoleforce_Bext}
  \vec{F}^{dd}_{i,j} = \frac{3\mu_0 m^2}{4\pi r^4}
\left[ 2(\hat{x}_{i,j} \times \hat{B}) \times \hat{B} - 2\hat{x}_{i,j} + 5\hat{x}_{i,j}|\hat{x}_{i,j} \times \hat{B}|^2 \right],
\end{equation}

where $m = \frac{\chi V B_{\text{ext}}}{\mu_0}$, $\hat{B} = \vec{B}_{\text{ext}}/B_{\text{ext}}$ is the unit vector along the external field, and $B_{\text{ext}} = |\vec{B}_{\text{ext}}|$. 

This force only applies for one particle, to obtain the total dipole interaction with all particles of the system, we should add them up, pair by pair:

\begin{equation}
  F^{dd}_i = \sum^{n}_{i \neq j} F^{dd}_{i,j}.  
  \label{eq:dipolesum}
\end{equation}

Let's remember that one important rule for the universe is that, one body cannot occupy the same space as other, therefore, we should modelate a respulsion force that simulates that behavior. Fortunately, there is a potential called Weeks-Chandler-Andersen potential (WCA) that follow the same principle as the Lennard-Jones potential but with a cuttof \cite{hess1999augmented}:

\begin{equation}
  U_{i,j}^{WCA} = \begin{cases} 
    4\epsilon\left[ \left( \frac{\sigma}{r_{i,j}}\right)^{12} - \left( \frac{\sigma}{r_{i,j}}\right)^6 + \epsilon\right] \quad &r_{i,j} \leq r_c \\
    0 \quad & r_{i,j} \geq r_c
  \end{cases}
  \label{eq:wcapotential}
\end{equation}

where $\sigma$ is the van der Waals radio, it is the point where the energy between the two particles is zero, $\epsilon$ is the well depth, since WCA does not have an attraction interaction, a correction factor of $\epsilon$ is added, and $r_c = 2^{1/6}\sigma$. This however, gives us the enegy between particles, and we need the force, then we apply:

\begin{equation}
 F = - \nabla U, 
  \label{eq:negativegradient}
\end{equation}

getting then

\begin{equation}
  F_{i,j}^{WCA} = \begin{cases} 
    48\epsilon\left[ \left( \frac{\sigma}{r_{i,j}}\right)^{12} - 0.5\left( \frac{\sigma}{r_{i,j}}\right)^6\right]\left[ \frac{1}{r_{i,j}}\right] \quad &r_{i,j} \leq r_c \\
    0 \quad & r_{i,j} \geq r_c
  \end{cases}.
  \label{eq:wcaforce}
\end{equation}

and verifying the interaction between each particle, we obtain:

\begin{equation}
  F^{collision}_i = \sum^{n}_{i \neq j} F^{WCA}_{i,j}.  
  \label{eq:wcasum}
\end{equation}

We want to see how the positions of the particles evolve over time with certain initial conditions, and this equations only compare velocities, this is why we need an integration method.

\section{Velocity Verlet}
%%%%%%%%

The Velocity Verlet algorithm is derived from the standard Verlet method by explicitly incorporating velocity terms \cite{verlet1967computer, chambliss2020magnetic}. We begin with the Taylor series expansions of position.

\subsection{Taylor Series Expansions}

The forward Taylor expansion for position is:
\begin{equation}
    x(t+\Delta t) = x(t) + v(t)\Delta t + \frac{1}{2}a(t)\Delta t^2 + \frac{1}{6}b(t)\Delta t^3 + O(\Delta t^4)
    \label{eq:taylor_forward}
\end{equation}

The backward Taylor expansion is:
\begin{equation}
    x(t-\Delta t) = x(t) - v(t)\Delta t + \frac{1}{2}a(t)\Delta t^2 - \frac{1}{6}b(t)\Delta t^3 + O(\Delta t^4)
    \label{eq:taylor_backward}
\end{equation}
where $b(t)$ represents the third derivative (jerk), and $O(\Delta t^4)$ represents an error of magnitude $\Delta t^4$.

\subsection{Original Verlet Algorithm}

Adding equations (\ref{eq:taylor_forward}) and (\ref{eq:taylor_backward}) cancels the velocity and jerk terms:
\begin{equation}
    x(t+\Delta t) + x(t-\Delta t) = 2x(t) + a(t)\Delta t^2 + O(\Delta t^4)
\end{equation}

Solving for $x(t+\Delta t)$ gives the original Verlet algorithm:
\begin{equation}
    x(t+\Delta t) = 2x(t) - x(t-\Delta t) + a(t)\Delta t^2 + O(\Delta t^4)
    \label{eq:verlet_original}
\end{equation}

\subsection{Velocity Calculation}

Subtracting equation (\ref{eq:taylor_backward}) from (\ref{eq:taylor_forward}) gives the velocity:
\begin{equation}
    x(t+\Delta t) - x(t-\Delta t) = 2v(t)\Delta t + O(\Delta t^3)
\end{equation}
\begin{equation}
    v(t) = \frac{x(t+\Delta t) - x(t-\Delta t)}{2\Delta t} + O(\Delta t^2)
    \label{eq:verlet_velocity}
\end{equation}

\subsection{Velocity Verlet Formulation}

The Velocity Verlet algorithm provides a more convenient form that explicitly tracks velocities. Starting from the forward Taylor expansion (equation \ref{eq:taylor_forward}):
\begin{equation}
    x(t+\Delta t) = x(t) + v(t)\Delta t + \frac{1}{2}a(t)\Delta t^2
    \label{eq:vv_position}
\end{equation}

To find the velocity at $t+\Delta t$, we use the average acceleration:
\begin{equation}
    v(t+\Delta t) = v(t) + \frac{a(t) + a(t+\Delta t)}{2}\Delta t
    \label{eq:vv_velocity_simple}
\end{equation}

A more numerically stable implementation uses a half-step approach:
\begin{align}
    v(t+\tfrac{1}{2}\Delta t) &= v(t) + \frac{1}{2}a(t)\Delta t \\
    a(t+\Delta t) &= \frac{F(x(t+\Delta t))}{m} \quad \text{(Compute new acceleration)} \\
    v(t+\Delta t) &= v(t+\tfrac{1}{2}\Delta t) + \frac{1}{2}a(t+\Delta t)\Delta t
    \label{eq:vv_velocity_halfstep}
\end{align}

%%%%%%%%

\section{Data analysis}

The simulations of this thesis, were done with an open-source software called LAMMPS (Large-scale Atomic/Molecular Massively Parallel Simulator) developed by Sandia National Laboratories. Its advantage is the high efficiency to run simulatios in parallel, reducing the time required to perform multiple calculations \cite{LAMMPS}. We used a standard version of LAMMPS but created custom input scripts using a Python library. These scripts will atumate the process of the input scritpts that will tell LAMMPS everything about the parameters of our simulation, such as the size of the system, the physical conditions, and how long to run for. After the simulations, we used additional custom scripts with the help of a Python library called Pandas, to analyze the results quickly and efficiently.
