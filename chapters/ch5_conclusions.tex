\part{Conclusions}
\label{part:conclusions}

\chapter{Conclusions}

Locomotion at the microscale presents unique challenges: viscosity dominates, inertia vanishes, and random thermal forces constantly disrupt order. Yet, as demonstrated throughout this thesis, these same constraints can be harnessed to induce controlled motion when coupled with asymmetry and periodic driving. The central goal of this work was to determine whether externally driven, yet passive, paramagnetic colloids could generate a net torque on an asymmetric ratchet through the interplay of geometry, dipolar interactions, and a precessing magnetic field.

Our results provide clear evidence that such rectification is achievable under specific conditions.
We first reviewed the landscape of microscale transport—ranging from natural microswimmers to artificial active particles—and highlighted the limitations associated with biologically or chemically powered systems. As an alternative, magnetically driven environments offer controllability, stability, and the possibility of working with purely passive particles. This motivated our exploration of paramagnetic colloids confined around a solid ratchet and driven by a conical magnetic field.

Using Brownian and molecular dynamics simulations, we examined the angular displacement and angular velocity of a rigid ratchet interacting with many dipolar colloids. The key finding is the emergence of a frequency window between approximately 3.25 and 8 Hz in which the system exhibits a persistent, statistically significant positive angular velocity. More importantly, within this  region, the rotation remains remarkably stable across fluctuations. This demonstrates partial rectification of motion driven solely by passive particles subjected to an external field—directly supporting the hypothesis that collective neighbor-exchange dynamics can generate net torque on an asymmetric object.

This result is particularly striking because it shows that the system can produce directed mechanical motion without requiring active particles, substrate patterning, or chemical gradients. In other words, it is possible to obtain useful work from a fully passive medium when immersed in an asymmetric geometry under external modulation. Additional analyses, including moving-mean filtering of individual trajectories and histograms at representative low (1 Hz) and high (5 Hz) frequencies, confirmed the validity and robustness of this behavior.

The study also uncovered an unexpected feature: rotating the ratchet geometry by 180° did not reverse the direction of rotation. The ratchet walls were mirrored, yet the torque direction persisted. This observation raises an open question about whether rotation is primarily dictated by the external field’s precession or if more subtle geometric cues influence the collective dynamics. This insight points toward a deeper symmetry-driven mechanism deserving further investigation.

\paragraph{Future work}

The findings of this thesis open several promising research directions: 
\begin{itemize}
  \item \textbf{Geometric optimization}: Systematically varying the ratchet radius, spike number, wall asymmetry, and cavity shape to enhance rotational efficiency.
  \item \textbf{Particle-density effects}: Exploring how higher or lower colloid packing influences neighbor-exchange and torque generation.
  \item \textbf{Controlled reversal}: Studying wheter field parameters, specifically direction of rotation, can reverse the rotation direction. 
  \item \textbf{Collective dynamics}: Exploring how the system behaves when multiple solid ratchets are introduced, arranged in either symmetric or asymmetric patterns. This includes studying arrays with alternating spin-up and spin-down orientations, as well as ratchets with different numbers of teeth. Such configurations may reveal emergent collective effects, such as synchronized rotation, mutual locking, or cooperative torque amplification, that are not present in single-ratchet systems.
\end{itemize}

In summary, this thesis demonstrates that passive paramagnetic colloids, when driven by a precessing magnetic field, can generate a measurable and reproducible torque on an asymmetric ratchet. The existence of a well-defined frequency window leading to stable rotation is a key result that validates the initial hypothesis and paves the way toward designing simple, robust microscale mechanical devices powered by collective Brownian dynamics.
