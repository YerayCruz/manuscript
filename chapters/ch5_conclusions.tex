\part{Conclusions}
\label{part:conclusions}

\chapter{Conclusions}

Locomotion at the microscale presents unique challenges: viscosity dominates, inertia vanishes, and random thermal forces constantly disrupt order. However, as shown throughout this work, these same constraints can be harnessed to induce controlled motion through asymmetry and periodic forcing. This thesis examined how ratchet mechanisms can rectify the otherwise random dynamics of microscopic particles, with particular emphasis on optically structured potential landscapes. The results highlight how geometry, driving frequency, and field symmetry collectively shape transport phenomena in active and passive systems.

We also discussed advances in artificial active particles, while emphasizing natural systems. These biological microswimmers demonstrate how energy can be converted into motion and even into useful work, giving rise to micromachines capable of transporting objects or generating rotation through chemical or biological activity. However, these systems face practical limitations: they are prone to biological degradation and require a continuous source of energy to remain active.

As an alternative, we discussed magnetically driven environments, focusing on spherical paramagnetic colloids driven by an external magnetic field. Experimental studies have already demonstrated that it is possible to rectify the motion of such particles, typically using ferrite garnet films to generate the required asymmetry. Yet, there also exist examples where directed motion is achieved without these substrates, relying only on a conical magnetic field.

In this thesis, we explored the effects of a conical external magnetic field on a solid ratchet immersed in paramagnetic spherical colloids suspended in water and confined geometry. The system was studied using Brownian and molecular dynamics simulations. We analyzed physical quantities such as angular displacement and angular velocity, as they are key indicators of motion rectification.

Our analysis suggests that in the frequency range of 3.25–8 Hz there is a noticeable increase in angular velocity, which remains roughly constant—though with some fluctuations—between 3.75 and 7.75 Hz. This indicates partial rectification of motion. Although the effect is not strong enough to produce complete rotations within short timescales, it opens a new line of research to enhance the efficiency of this type of system. The same behavior was observed for all four ratchet configurations studied. Interestingly, when the ratchet geometry was rotated by 180°, the overall direction of rotation remained unchanged, even though the ratchet walls were mirrored. This raises an open question about whether the rotation can be controlled through geometry or if it depends entirely on the external magnetic field.
To ensure the accuracy of these results, we performed additional analyses on individual datasets using a moving mean filter, and histogram for the lowest (1 Hz) and highest (5 Hz) frequencies. These results confirmed the overall behavior observed in the averaged data and validated our main findings.

In summary, we explored how it is possible to obtain work from passive environments when induced in a assymetric potentials in non-symmetric objects. This thesis opens to discussion about how other parameters of the geometry of the ratchet can effectively affect the rotation of it, such as the radius, wall length, a lower or bigger amount of \textit{spikes}, a bigger amount of colloids, different colloids radius. 
