\part{Introduction}
\label{part:introduction}

%======================================================================
\chapter{Introduction}
\label{ch:introduction}

\markright{Introduction}
%======================================================================
The study of motion at the microscale lies at the intersection of physics, chemistry, and biology, with implications for both fundamental understanding and technological applications. In this regime, inertia becomes negligible and viscous forces dominate, while random fluctuations from thermal noise play a significant role. Understanding how to control, rectify, and harness such motion has therefore become a central challenge in soft matter and statistical physics~\cite{purcell2014life,einstein1906theory}. Beyond theoretical curiosity, these ideas underpin innovations in microfluidic devices, targeted drug delivery, and nanoscale engines.

Nature already offers remarkable examples of how microscopic systems can convert energy into directed motion. Within cells, molecular motors such as kinesin and dynein transport cargo along cytoskeletal filaments. On larger biological scales, microorganisms like \textit{E. coli}, sperm cells, and algae have evolved efficient strategies to swim through highly viscous environments~\cite{howard2002mechanics, vale2003molecular, marchetti2013hydrodynamics}. The study of these self-propelled entities has led to the emergence of the field of active matter, which explores collections of self-driven units operating far from equilibrium~\cite{ramaswamy2010mechanics, needleman2017active, bechinger2016active}. In recent years, this concept has expanded to include collective and emergent behaviors, such as topological flows in active materials~\cite{shankar2022topological}.

To channel and control this active motion, researchers have drawn inspiration from another concept: ratchets. Ratchets are asymmetric structures or potentials that convert random fluctuations into net directional motion, a principle famously illustrated by Feynman’s thought experiment. Their study has led to the development of Brownian motors, systems that harness thermal or active noise for controlled transport~\cite{julicher1997modeling, reimann2002brownian}. More recently, these principles have been extended to active matter, where active Brownian particles can be rectified into persistent directional flows~\cite{fiasconaro2008active, reichhardt2017ratchet, rein2023force}.

A striking realization of this idea was demonstrated by Di Leonardo~\cite{di2010bacterial}, who showed that asymmetric microgears immersed in dense bacterial suspensions could rotate spontaneously, powered solely by the activity of \textit{E. coli}. These bacterial ratchets combine nonequilibrium activity with geometric asymmetry to generate mechanical motion without external input. Subsequent studies have revealed that bacterial collisions with curved microstructures can exert measurable forces, and that the curvature of the structure itself can modulate its rotation speed~\cite{pellicciotta2025wall}. Together, these results highlight how geometric design and collective active behavior can be harnessed for autonomous transport and mixing at the microscale.

Even with these advances, there are still open problems. Active matter is powerful but not always easy to control or keep stable, since living systems depend strongly on their environment~\cite{bechinger2016active}. Passive colloids are easier to handle in experiments but need external driving to move in a directed way. Magnetically driven colloids are promising in this respect because their interactions can be tuned, they are reversible, and they can be controlled externally. Experiments have shown magnetic colloidal carpets that propel and carry cargo~\cite{martinez2015magnetic}, and other studies have revealed enhanced diffusion, synchronization, and bidirectional transport under confinement~\cite{tierno2012depinning, straube2014tunable, massana2020emergent, ostinato2024magnetically}. Still, the interplay between confinement, collective effects, and external driving is not fully understood.

This thesis looks at these questions by using numerical simulations of paramagnetic colloids under magnetic fields, with emphasis on confinement and ratchet effects. Simulations are useful because they allow us to test conditions that are hard to achieve in experiments and to separate the roles of noise, interactions, and driving. By studying transport properties, collective effects, and symmetry breaking, we aim to clarify how random microscopic motion can be rectified into useful work. The results connect to basic questions in nonequilibrium systems and may also guide the design of micro- and nanoscale devices, where controlled transport is essential.

%%%%%
This work is guided by a central question: under what conditions can externally driven, yet passive, paramagnetic colloids generate rectified motion capable of performing mechanical work?
Our hypothesis is that the interplay between precessing magnetic fields, confinement, and dipolar interactions produces dynamical regimes, particularly the neighbor-exchange regime, in which fluctuations and symmetry breaking combine to yield a net torque on an asymmetric ratchet. By systematically varying the field frequency, particle packing, and ratchet geometry, we evaluate this hypothesis by measuring the resulting rotational response of the solid ratchet across multiple simulations and computing the mean angular frequency as a quantitative indicator of rectification.
%%%%%

The thesis is organized as follows. Part~\ref{part:background} reviews the theoretical background, including motion at low Reynolds number, Brownian noise, ratchet mechanisms, and the difference between active and passive matter. Part~\ref{part:methods} describes the simulation methods, with focus on stochastic molecular dynamics. Part~\ref{part:results} presents the main findings on confined colloidal dynamics and rectification. Finally, Part~\ref{part:conclusions} summarizes the contributions and suggests future work.
