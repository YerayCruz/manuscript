


\part{Introduction}
\label{part:introduction}

%======================================================================
\chapter{Introduction}
\label{ch:introduction}

\markright{Introduction}
%======================================================================

The study of motion at the microscale lies at the intersection of physics, chemistry, and biology, and has profound implications for both fundamental science and applied technologies. At these scales, inertia is negligible and viscous forces dominate, while random fluctuations arising from thermal noise strongly influence dynamics. Understanding how to control, rectify, and exploit such motion is a central question in modern soft matter physics and statistical mechanics~\cite{purcell2014life, einstein1906theory}. Beyond its theoretical significance, it also bears practical importance in the design of microfluidic devices, targeted drug delivery systems, and nanoscale engines.

Nature offers numerous examples of how microscopic systems can efficiently convert energy into motion. Molecular motors such as kinesin and dynein transport cellular cargo, while microorganisms like \textit{E. coli}, spermatozoa, and algae have evolved diverse strategies to propel themselves in viscous environments~\cite{howard2002mechanics, vale2003molecular, marchetti2013hydrodynamics}. These active matter systems provide inspiration for artificial analogues, where researchers attempt to harness fluctuations or external fields to generate directed transport. A particularly powerful framework is that of ratchets: asymmetric structures or potentials that rectify random motion into net work, recalling Feynman’s famous thought experiment. This idea has been realized both in theoretical models and experimental systems, giving rise to the field of Brownian motors~\cite{julicher1997modeling, reimann2002brownian}.

Recent experimental advances have demonstrated that ratchet concepts can be realized using living systems. For example, Di Leonardo et al. (2010) showed that asymmetric microgears immersed in a dense bacterial suspension could spontaneously rotate, powered only by the metabolic activity of \textit{E. coli}~\cite{di2010bacterial}. Such bacterial ratchet motors provide a striking example of how nonequilibrium activity and geometric asymmetry can be combined to produce directed motion.

Despite these advances, several challenges remain. Active matter is powerful but often difficult to control and maintain, as living systems are sensitive to environmental conditions~\cite{bechinger2016active}. Passive colloidal systems, on the other hand, are experimentally more robust but require clever external driving strategies to achieve directed motion. Magnetically driven colloids are especially promising because they combine controllability, reversibility, and tunability of interactions. Confinement and geometry further enrich their dynamics, giving rise to phenomena such as enhanced diffusion, synchronization, and bidirectional currents~\cite{tierno2012depinning, straube2014tunable, ostinato2024magnetically}. Yet, a complete understanding of how these factors interplay to produce ordered motion from stochastic fluctuations is still lacking.

This thesis contributes to addressing this gap by employing numerical simulations to study paramagnetic colloids under external magnetic fields, with a particular focus on confinement and ratchet-like behavior. Simulations offer a controlled platform to probe conditions that are difficult to realize experimentally and to disentangle the roles of noise, interactions, and driving. By analyzing transport properties, collective effects, and symmetry-breaking mechanisms, this work aims to shed light on the principles by which random microscopic motion can be converted into useful work. Beyond advancing our theoretical understanding of nonequilibrium systems, these results may inform the design of new micro- and nanoscale devices, where efficient control of particle motion is a central requirement.

\newpage
