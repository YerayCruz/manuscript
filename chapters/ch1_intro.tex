


\part{Introduction}
\label{part:introduction}

%======================================================================
\chapter{Introduction}
\label{ch:introduction}

\markright{Introduction}
%======================================================================
The study of motion at the microscale sits between physics, chemistry, and biology, and has consequences both for basic science and for applications. At these scales, inertia can be ignored and viscous forces dominate. Random fluctuations from thermal noise also play an important role. How to control, rectify, and use this motion is one of the main questions in soft matter and statistical physics~\cite{purcell2014life,einstein1906theory}. Besides the theoretical interest, these ideas are useful for the design of microfluidic devices, targeted drug delivery, and even nanoscale engines.

Nature already provides many examples of how small systems can turn energy into motion. Inside cells, molecular motors like kinesin and dynein move cargo. Microorganisms such as \textit{E. coli}, sperm cells, and algae have developed strategies to swim in viscous environments~\cite{howard2002mechanics, vale2003molecular, marchetti2013hydrodynamics}. These observations gave rise to the field of \textit{active matter}, which studies collections of self-driven units operating out of equilibrium~\cite{ramaswamy2010mechanics, needleman2017active, bechinger2016active}. The concept has expanded in recent years to cover emergent effects, such as topological flows in active materials~\cite{shankar2022topological}.

One useful way to think about directed motion at this scale is through ratchets. These are asymmetric structures or potentials that turn random fluctuations into net motion, similar to Feynman’s ratchet thought experiment. Ratchets have been studied in both theory and experiments, leading to the broader field of Brownian motors~\cite{julicher1997modeling, reimann2002brownian}. More recently, these ideas have been extended to active matter, where active Brownian particles can be rectified into directional transport~\cite{fiasconaro2008active, reichhardt2017ratchet, rein2023force}.

Di Leonardo \textit{et al.} (2010) harnessed the movement of living cells and showed that asymmetric microgears immersed in dense bacterial suspensions could rotate spontaneously, powered only by the activity of \textit{E. coli}~\cite{di2010bacterial}. These bacterial ratchets combine nonequilibrium activity with geometric asymmetry to produce motion. Such systems point to possible uses in autonomous transport and mixing at the microscale. Also it has been studied how bacterium collisions with curved-geometry microstructures can exert net movement in these structures and how the curvature of the surface can modify the speed of the structure~\cite{pellicciotta2025wall}.

Even with these advances, there are still open problems. Active matter is powerful but not always easy to control or keep stable, since living systems depend strongly on their environment~\cite{bechinger2016active}. Passive colloids are easier to handle in experiments but need external driving to move in a directed way. Magnetically driven colloids are promising in this respect because their interactions can be tuned, they are reversible, and they can be controlled externally. Experiments have shown magnetic colloidal carpets that propel and carry cargo~\cite{martinez2015magnetic}, and other studies have revealed enhanced diffusion, synchronization, and bidirectional transport under confinement~\cite{tierno2012depinning, straube2014tunable, ostinato2024magnetically}. Still, the interplay between confinement, collective effects, and external driving is not fully understood.

This thesis looks at these questions by using numerical simulations of paramagnetic colloids under magnetic fields, with emphasis on confinement and ratchet effects. Simulations are useful because they allow us to test conditions that are hard to achieve in experiments and to separate the roles of noise, interactions, and driving. By studying transport properties, collective effects, and symmetry breaking, we aim to clarify how random microscopic motion can be rectified into useful work. The results connect to basic questions in nonequilibrium systems and may also guide the design of micro- and nanoscale devices, where controlled transport is essential.

The thesis is organized as follows. Part~\ref{part:background} reviews the theoretical background, including motion at low Reynolds number, Brownian noise, ratchet mechanisms, and the difference between active and passive matter. Part~\ref{part:methods} describes the simulation methods, with focus on stochastic molecular dynamics. Part~\ref{part:results} presents the main findings on confined colloidal dynamics and rectification. Finally, Part~\ref{part:conclusions} summarizes the contributions and suggests future work.

\newpage
