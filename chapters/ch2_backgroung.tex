



\part{Background}
\label{part:background}

%======================================================================
\chapter{Swimming at the mesoscale}
\label{ch:swimming at the mesoscale}

\markright{Swimming at the mesoscale}
%======================================================================

\section{Low Renolds number}

At the mesoscale, where objects such as bacteria and colloidal particles operate, the physical world is governed by a regime in which viscous forces dominate over inertial ones. This regime is characterized by a small Reynolds number (Re), a dimensionless quantity that compares inertial to viscous effects. Therefore, the force applied at that moment will describe the movement or displacement performed, not deppending on any past force, this is a characteristic of an overdamped system. In his seminal lecture, Life at Low Reynolds Number, Purcell highlighted the surprising and often counterintuitive behaviors that emerge in such environments \cite{purcell2014life}. For instance, time-reversible motion — common at macroscopic scales — is ineffective for propulsion at low Re, necessitating non-reciprocal strategies like flagellar rotation or body undulation. 

This physical constraint fundamentally shapes how microorganisms swim and how artificial microswimmers must be designed. Subsequent work by Lauga and Powers [Lauga Powers, 2009] expanded on Purcell's insights by examining the fluid dynamics of various propulsion mechanisms, including bacterial run-and-tumble behavior and the synchronization of flagella. These concepts are directly relevant to systems explored in this thesis, where both biological and magnetically driven entities are used to induce motion in passive structures at the microscale.



\newpage
